%%% Research Diary - Entry
%%% Template by Mikhail Klassen, April 2013
%%% 
\documentclass[14pt]{extarticle}

%\newcommand{\workingDate}{\textsc{2013 $|$ January $|$ 01}}
\newcommand{\userName}{Arif Hossain} \newcommand{\institution}{Dhaka
    Universeity}
%\usepackage{researchdiary_png}
\usepackage{extsizes} \usepackage[csdisplay=true]{csquotes}
% To add your univeristy logo to the upper right, simply upload a file named
% "logo.png" using the files menu above.

\begin{document}
%\univlogo

\title{}

{\Huge June 1 2015 }\\[5mm]

{\Large  Elite Politics}\\[3mm] One of the important observation has to be the
disguise of nationalist politics. Nationalist politics claims representation.
But hardly it can claim it. But if we're ready concede representation, we have
to analyze their empirical basis more. \par

What does nationalist politics do in face of class tension  inside its
constituency?\par

It's true that there existed class tension in every nationalist politics. What
happens, nationalist discourse either try to reconcile with it by conceding
some to the lower masses. But if the lower masses themselves get some
representation and try to have their way, then the true nature  of their their
core convictions comes out. \par

I'm going to record here two parallel example. One from Bengal politics, and
another from  western provinces. One involves Congress, another involves
League.\par When farmer and tenants organized themselves and agitated in 1930s
for their long standing grievences from their treatment by the hands of
jamindars, they voted to have their way. The Huq ministry came about by
promising an end to these grivences specifically.The League which did bad in
1937 election did ran with a communal line. The most defining moment of the
election came about when the old hog Nazimuddin got roundly defeated by Huq in
Potuakhali. People had spoken. They wanted to end the zamindari regime which
held them hostages for centuries. Because of the pressure from constituency Huq
tried to push legislation to tighten control over zamindari system. He was
venomously opossed by hindu establishment and curiously also he was opoosed by
Muslim elites. Because these elites were only thinking of their vested interest
and they were not ready to let lower classes muddle their previleges. The fear
of lower class agitation ended up fierce move to the right and expulsion of all
radical and in places mildly radical people out of the Congress fold. This is
extraordinary when we see Hindu elites gleefully backing new Hindu Mohashobha.
How convinient. It is also pressing that Marx after-all seems to be right about
the significance of class relations. Same thing happened to League in western
provinces which made many calculated move to keep out agitational politics with
mass base.\par

The irony in all this dishonest manuvering is, it is oppressed masses felt the
true consequences of the move to the right.

It has to be true the rise of communalism in 1930s and onward, has to be
understood in class relations existed between different parties. Also there are
issues of cultural fascism. But i have an understanding that it was class
relations which played the most important part.

{\Large Random though }\\[3mm]


\blockquote{\small } \end{document}
