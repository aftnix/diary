%%% Research Diary - Entry
%%% Template by Mikhail Klassen, April 2013
%%% 
\documentclass[14pt]{extarticle}

%\newcommand{\workingDate}{\textsc{2013 $|$ January $|$ 01}}
\newcommand{\userName}{Arif Hossain}
\newcommand{\institution}{Dhaka Universeity}
%\usepackage{researchdiary_png}
\usepackage{extsizes}
\usepackage[csdisplay=true]{csquotes}
% To add your univeristy logo to the upper right, simply
% upload a file named "logo.png" using the files menu above.

\begin{document}
%\univlogo

\title{}

{\Huge June 1 2015 }\\[5mm]

One of the important observation has to be the disguise of nationalist politics. Nationalist politics claims representation. But hardly it 
can claim it. But if we're ready concede representation, we have to analyze their empirical basis more. \par

What does nationalist politics do in face of class tension  inside its constituency?\par

It's true that there existed class tension in every nationalist politics. What happens, nationalist discourse either try to
reconcile with it by conceding some to the lower masses. But if the lower masses themselves get some representation and 
try to have their way, then the true nature  of their their core convictions comes out. \par

I'm going to record here two parallel example. One from Bengal politics, and another from  western provinces. One involves
Congress, another involves League.\par
When farmer and tenants organized themselves and agitated in 1930s for their long standing grievences from their treatment
by the hands of jamindars, they voted to have their way. The Huq ministry came about by promising an end to these grivences
specifically.The League which did bad in 1937 election did ran with a communal line. The most defining moment of the
election came about when the old hog Nazimuddin got roundly defeated by Huq in Potuakhali. People had spoken. They
wanted to end the zamindari regime which held them hostages for centuries. Because of the pressure from constituency Huq tried to push legislation

\blockquote{\small 
\end{document}
